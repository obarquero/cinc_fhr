\section{Introduction}

Perinatal hypoxia is a fetus and newborn disease due to the lack of tissues oxygenation. Although it can occur in earlier gestation phases, childbirth and immediate neonatal hours are the fundamental risk periods.

Perinatal hypoxia severity spectrum conveys very mild cases (only requiring neonatal resuscitation with environmental oxygen), more serious cases needing intubation and acidosis correction with bicarbonate (reanimation types V and VI), and critical cases that can cause perinatal death or serious sequels, such as brain or adrenal hemorrhage, necrotizing enterocolitis, delayed neurological development, mental handicap, seizures (West syndrome) or cerebral palsy~\cite{Leuthner2004,Morales2011}. Diagnosis is performed at the birth time by evaluating the cardio-respiratory depression and the muscle tone. The severity of the hypoxia is commonly quantified using the Apgar Score~\cite{Apgar1953,casey2001continuing}, with a score lower than 7 at five minutes after delivery being considered as pathological, which is  usually confirmed with gas analysis of the umbilical cord, whereas low pH values  evidence  metabolic acidosis. Typical values considered for diagnosis are pH $\leq$ 7.05 or pH $\leq$ 7, and intrapartum pH values $\leq$  7.20  are considered pathological in terms of risk of perinatal hypoxia.

Continuous electronic fetal monitoring, also known as Cardiotocography (CTG), was developed around 1960~\cite{Hon1958,Hammacher1968} and consists of the simultaneous evaluation of the Fetal Heart Rate (FHR) and the uterine activity.  After CTG generalization, two relevant signs of suspicious fetal hypoxia were recognized, namely, the late decelerations of the FHR in relation to uterine contractions, and the FHR variability decrease~\cite{Low1999}. Although visual interpretation of CTG has an acceptable sensitivity for  risk of hypoxia detection (specially in pathological traces), the specificity still is low (specially for suspicious traces), and it requires the confirmation with invasive pH determination of scalp blood of the fetus, which is technically cumbersome and not always feasible~\cite{Tasnim2009}. When considering the risk of hypoxia, gynaecologists indicate interventions (cesareans, forceps, and vacuum extraction) more often than necessary~\cite{Tasnim2009}, hence increasing sensitivity at the expense of specificity. In addition, visual assessment of bradycardias and late decelerations is simple, whereas visual assessment of the loss of variability is not and even it varies among observers, representation type (computer display or paper), or cardiotocograph model~\cite{Ayres-de-Campos1999,Bernardes1997,Santo2012}.

\myorange{cut the clutter in the previous paragraphs.}

\myorange{Several nonlinear indices has been proposed to assess the complexity from the FHR signal~\cite{}}

\myorange{We aim assess the change in the complexity of the physiological mechanism tha control the FHR due to hypoxia before the partum. We assess the FHR complexity using three differente nonlinea measures, Time Irreversibility, SampEn and Permutation Entropy in 32 foetal recordings (15 controls).}
%
\myorange{The draw of the paper is as follows. Section~\ref{sec:system} describes  the considered alternatives for the elements of the detection system: signal segmentation, feature extraction, signal similarities computation, feature selection and classification. Then, Section~\ref{sec:experiments} experimentally demonstrates the capability of NCD both for classification of raw signals and for extending the capabilities of conventional analysis in a real FHR dataset. Finally, Section~\ref{sec:discussion} discusses the main advantages of the proposed methodology over other alternatives and concludes the paper.}