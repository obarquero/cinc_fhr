\begin{abstract}
Heart Rate Turbulence (HRT) is a relevant cardiac risk stratification criterion. It is usually assess by means of turbulence slope ($TS$) and turbulence onset ($TO$) parameters. HRT is known to be affected by several physiological factors, mainly heart rate (HR) and coupling interval ($CI$). The physiological hypothesis accepted is the baroreflex source of the HRT. However, several studies showed different results about the relationship between CI and HRT parameters. Our aim was to propose a complete LASSO model using $CI$ and sinus cardiac length ($SCL$), their powers and an interaction term as explanatory variables to account for the physiological dynamic of the $TS$ parameter. 

We used a database of 61 recording holters from acute myocardial infarction (AMI) patients. The database was split into two groups; low-risk patients \mbox{($TS > 2.5\ \&\ TO < 0$)}, and high-risk patients \mbox{($TS < 2.5\ \&\ TO > 0$)}. We performed a feature analysis by means of the LASSO paths, in which the regularization parameter is changed from very high values, where all weights of the explanatory variables are zero, to small values were all the weights are different from zero. 

The first variable activated, with a coefficient different from zero, was $SCL$ on low-risk patients and the two following where related to $CI$. Whereas the first variable activated on high-risk patient was $CI$ and the two following were relate to $SCL$. Results from LASSO paths suggest that the influence of physiological variables on HRT is broken on AMI high-risk and completely different from low-risk. Also, the features selected by LASSO model on AMI low-risk are in agreement with the hypothesis of a baroreflex source of the HRT, in which $SCL$ is the most important variable, and $CI$ has a negative correlation with $TS$.
\end{abstract}
