\begin{abstract}
Background Perinatal hypoxia is a severe condition that may harm fetus organs permanently or even cause dead. When the fetus brain is partially deprived from oxygen, the control of the fetal heart rate (FHR) is affected.
 
Objective. We hypothesize that the complex physiological mechanisms of the FHR are perturbed under perinatal hypoxia. We propose measure entropy and time irreversibility of the FHR to quantify the loss in the complexity.
 
Materials and Methods. We estimated the complexity of the FHR signal using Sample Entropy (SampEn), Permutation Entropy (PE), and Time Irreversibility (TI). We compared the results with time (Short Time Variability, STV) and frequency domain (High Frequency Power, PHF) methods. We computed every one hour before delivery. FHR traces were preprocessed to remove artifacts. A database of 32 FHR recordings were acquired with cardiotochography, 15 controls and 16 cases. A case was declared whether: 1) the PH of the umbilical artery was $\leq$ 7.05; or 2) the APGAR score 5 minutes after delivery was $\leq$ 7 and a reanimation type III or greater was required. Resampling methods were used to establish the statistical differences.
 
Results. TI was significantly different for healthy and hypoxia fetuses (-0.38 +- 0.19 vs. -0.21+-0.37, p-value=0.063). Entropy indices were higher for healthy fetuses (SampEn:0.33+-0.12 vs 0.28 +- 0.09, p-value=0.11; PE:0.72+-0.04 vs 0.69+-0.07, p-value= 0.12). STV (3.23+-1.15 vs 3.45+-1.35, p-value=0.30) and PHF (0.40+-0.18 vs 0.43+-0.25, p-value=0.31) indices showed no differences.

Conclusions. Complexity measures of the FHR were different for healthy and hypoxia fetuses. These indices may help to early detect hypoxia with less invasive methods.

\end{abstract}
