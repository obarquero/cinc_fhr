\section{Discussion and Conclusions}
\label{sec:discussion}

\myblue{Several indices have been proposed to analyze FHR. The most common indices are based on time domain and frequency domain methods~\cite{Signorini:2003ee,vanLaar:2009dj}. Time domain methods aim to assess the long and short term variability of the FHR, whereas frequency domain methods aim to characterize the oscillatory contributions on the FHR. In many cases these indices are reduced to a single number obtained in the entire time series or to a collection of numbers obtained in 5-minute windows slides, which are again reduced to a few numbers like mean or standard deviation. }

We have proposed NCD as a similarity measure for FHR registers because it is able to exploit both linear and non-linear relations among records. We tried it with raw FHR records, time and frequency indices and moments signals extracted from sliding windows. We obtained better performance from the moments than from the raw records, which shows that the compressor is not able to extract all the relations in the data and preprocessing might help. Then, we have shown how to combine several moments (or other type of variable) by using a simple voting scheme and by summing the dissimilarity matrices, which provide overall better results in our case.

Main strengths of using NCD for comparing FHR registers or signals of  any statistic applied to signal windows are simplicity and generality. Other commonly used information-theoretical measures, like Approximate entropy~\cite{Pincus1991} or Sample entropy~\cite{Richman2000}, need to tune the embedding dimension and specially the tolerance, which is a continuous parameter; but there is no parameter to tune for our approach. Indeed, despite we have tried three compressors and two simple approaches to get a symmetric matrix  the proposed methodology can be straightforwardly used with a lzma compressor and the min approach to get symmetric NCD matrices.  In addition, there is no problem with the common signal losses, which is a problem to apply frequency-related methods as they need signal interpolation, which is not always possible. The similarity can always be computed independently of how the signal losses are addressed. 

\myblue{{\it Visual interpretation} values  basal FHR, its accelerations and decelerations in relation to uterine contractions, and beat-to-beat FHR variability~\cite{Ayres-de-Campos2010a,RajendraAcharya2006}.
The following signal types are considered clearly pathological (suspicious of hypoxia): late decelerations, whose nadir has a delay of at least 30 seconds with respect to the acme of contractions; maintained bradycardia;  low variability (less than 5 beats); and a ``sine''-rhythm, named after its wave-like appearance, which is characterized by a long-term variability but almost no variability in the short-term.}

\myblue{First, late decelerations are produced in the following manner. During uterine contraction, when the myometrium pressure exceeds the blood pressure of the intervillous space of the placenta, maternal circulation is interrupted and therefore ceases to carry oxygen to the fetal territory. In a well oxygenated fetus, this is not a problem since the fetus does not consume all the tissue-oxygen before the end of the contraction. In a fetus with poor oxygenation, on the other hand, the oxygen reserves are exhausted before the end of contraction and, particularly in the more sensitive heart cells (which act as a pacemaker), action-potential production mechanisms are delayed, which causes bradycardia.}

\myblue{And second, the decrease in variability is more difficult to explain. On the one hand,  the regulation of FHR, usually controlled by the vagus nerve, is stopped by cardiac and nervous system hypoxia. On the other hand,  it seems that there are less functionally active cells in the pacemaker as the hypoxia progresses. The regulatory system loses ``degrees of freedom'', which therefore becomes increasingly uniform and deterministic~\cite{Goldberger1987}.
}

\myblue{FHR signal can be measured in two ways, namely, external, by using a ultrasonic sensor that observes the Doppler effect; and internal, in which fetal electrocardiogram (ECG) is measured with an electrode in the fetus scalp. Uterine activity can also be measured in two ways: by  using a non-invasive pressure transducer in the mother's abdomen, or with an invasive intrauterine catheter pressure sensor}.

\myblue{Automatic analysis of CTG has also been proposed. In \cite{AMERWAHLIN2001}, automatic ST analysis (the ST segment connects the QRS complex and the T wave) combined with CTG was shown to increase the ability of obstetricians to identify hypoxia and to improve the perinatal outcome. In \cite{Ayres-de-Campos2010},  a clinical trial was proposed to assess whether computer analysis and alerts improves visual CTG monitoring. A system-identification approach was proposed \cite{Warrick2010} to model FHR and uterine activity as an input-output system,  reporting around 50\% sensitivity with 7.5\% of false positives at 1h and 40 minutes before delivery.  Time-frequency analysis and features from time-frequency space decomposition were sucessfully used in an animal study yielding  93\% sensitivity  and 98\% specificity\cite{Dong2014}.
Other non-invasive approaches have been proposed to complement CTG, such as Doppler velocimetry and pulse oximetry~\cite{Yang1998,Siristatidis2004,Mori2014}, or  near-infrared spectroscopy to measure cerebral metabolic rate of oxygen~\cite{Tichauer2009}. The performance of automatic approaches currently applied to hypoxia detection has still room for improvement both in accuracy (sensitivity and specificity) and in the time before birth where hypoxia is detected.}

\myblue{NCD technique was successfully used for clustering the fetuses of a multicentric study with the aim of identifying the abnormal ones~\cite{CostaSantos2006}. }

It is remarkable that, using sliding windows and NCD, both frequency indices and moments obtain the best accuracies in the  4 $\leftrightarrow$  3 hours interval whereas time indices obtain the best results in the  3 $\leftrightarrow$  2 hours interval. Our comparisons show that the commonly used Time and Frequency indices can be complemented by the moments, which are always applicable and do not suffer from signal loses. In addition, fetus movement might provide valuable information as we noted when analyzing raw signals (Table~\ref{tab:ncd:raw}), and when we observed the performance of $P_{LF} / (P_{MF}+P_{HF})$ index, which depends on fetus movement (Table~\ref{tab:ncd:time:freq:moments:5min}). Finally, by adding similarity matrices selected by FS a promising 0.88 accuracy is reached in the 4 $\leftrightarrow$  1 hours interval, which compares entire records and mimics the processing of a fetus during labor.

Practical implementation of this approach as an plugin of the available CTG systems is straightforward. We recommend to perform a careful selection and labeling of FHR records. Then, the number of cases in the knowledge database and processing capabilities must be balanced. For instance, the  analysis of an FHR record every minute against a knowledge database of 1000 patients is easily done in a normal PC using gzip as compressor.

%% JL: Creo que sería muy bueno que Carlos escribiera uno o dos párrafos de discusión y conclusiones.
 
%+ 1 párrafo o dos Discusión Carlos.

%\section{Conclusions}

The decision making during the labor is a difficult task for the gynecologist. It always should be intended to be as less invasive as possible, but, of course, ensuring fetal well-being and acting as soon as possible in case of  suspicion of fetal hypoxia. Our main contribution shows how the NCD analysis of the readily available FHR traces may help the gynecologists to make the correct decisions. We reach 88\% accuracy, which is a remarkable result if we take into account that we are actually identifying stressed fetuses 3 hours before delivery that were not detected by the gynecologist until a later stage.
This general methodology is also applicable to other time series classification problems and it is both simple to understand and simple to apply.

The results obtained in this study indicate that a further study with more patients should be performed to open the application of this type of FHR analysis of the fetus condition to the industry.